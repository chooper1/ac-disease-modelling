%%%%%%%%%%%%%%%%%%%%%%%%%%%%%%%%%%%%%%%%%
% Stylish Article
% LaTeX Template
% Version 2.1 (1/10/15)
%
% This template has been downloaded from:
% http://www.LaTeXTemplates.com
%
% Original author:
% Mathias Legrand (legrand.mathias@gmail.com) 
% With extensive modifications by:
% Vel (vel@latextemplates.com)
%
% License:
% CC BY-NC-SA 3.0 (http://creativecommons.org/licenses/by-nc-sa/3.0/)
%
%%%%%%%%%%%%%%%%%%%%%%%%%%%%%%%%%%%%%%%%%

%----------------------------------------------------------------------------------------
%	PACKAGES AND OTHER DOCUMENT CONFIGURATIONS
%----------------------------------------------------------------------------------------

\documentclass[fleqn,10pt]{SelfArx} % Document font size and equations flushed left

\usepackage[english]{babel} % Specify a different language here - english by default

%\usepackage{lipsum} % Required to insert dummy text. To be removed otherwise

%----------------------------------------------------------------------------------------
%	COLUMNS
%----------------------------------------------------------------------------------------

\setlength{\columnsep}{0.55cm} % Distance between the two columns of text
\setlength{\fboxrule}{0.75pt} % Width of the border around the abstract

%----------------------------------------------------------------------------------------
%	COLORS
%----------------------------------------------------------------------------------------

\definecolor{color1}{RGB}{0,0,90} % Color of the article title and sections
\definecolor{color2}{RGB}{0,20,20} % Color of the boxes behind the abstract and headings

%----------------------------------------------------------------------------------------
%	HYPERLINKS
%----------------------------------------------------------------------------------------

\usepackage{hyperref} % Required for hyperlinks
\hypersetup{hidelinks,colorlinks,breaklinks=true,urlcolor=color2,citecolor=color1,linkcolor=color1,bookmarksopen=false,pdftitle={Title},pdfauthor={Author}}

%----------------------------------------------------------------------------------------
%	ARTICLE INFORMATION
%----------------------------------------------------------------------------------------

\JournalInfo{Technical Report 2020-01} % Journal information
\Archive{} % Additional notes (e.g. copyright, DOI, review/research article)

\PaperTitle{Modelling SARS-Cov-2 at Acadia University} % Article title

\Authors{Holger Teismann\textsuperscript{1}*, Duane Currie\textsuperscript{2}, Franklin Mendivil\textsuperscript{1}, Coleman Hooper\textsuperscript{3}, Margaret Hopkins\textsuperscript{1}, Yifan Li\textsuperscript{4},  Richard Karsten\textsuperscript{1}} % Authors
\affiliation{\textsuperscript{1}\textit{Department of Mathematics and Statistics, Acadia University, Wolfville, Nova Scotia, Canada}} % Author affiliation
\affiliation{\textsuperscript{2}\textit{Institutional Research, Acadia University, Wolfville, Nova Scotia, Canada}} % Author affiliation
\affiliation{\textsuperscript{3}\textit{????}} % Author affiliation
\affiliation{\textsuperscript{4}\textit{Jodrey School of Computer Science, Acadia University, Wolfville, Nova Scotia, Canada}} % Author affiliation
\affiliation{*\textbf{Corresponding author}: holger.teismann@acadiau.ca } % Corresponding author

\Keywords{} % Keywords - if you don't want any simply remove all the text between the curly brackets
\newcommand{\keywordname}{Keywords} % Defines the keywords heading name

%----------------------------------------------------------------------------------------
%	ABSTRACT
%----------------------------------------------------------------------------------------

\Abstract{
We are a group of professors, \ed{a super-hero} and students who are building mathematical and computational models with the goal of informing decisions regarding policies aimed at preventing, detecting and containing a possible outbreak of COVID-19 at Acadia.
Our primary model is based on simulating all of the individuals at Acadia and their interactions.
A strong feature of our effort is the availability of (appropriately anonymized) historical data on class locations, class enrollments, residence room assignments and other similar details of the day-to-day activities of people at Acadia.
This greatly increases the realism of our model and also makes it very specific to the situation at Acadia.\\[5pt]
Using our model we can run simulations of many different combinations of interventions and policy decisions and thereby obtain a quantitative idea of their relative importance in controlling COVID-19.
\emph{We welcome suggestions for combinations to experiment with. }\\[5pt]
The result of any given simulation run depends on many different random choices since each interaction between an infective individual and a susceptible individual only results in a new infection with a certain probability.
Thus repeating the simulation many times under the same scenario gives statistical information about the range of possible outcomes.\\[5pt]
\emph{We do not claim that the actual numerical values from any simulation are precisely accurate predictions of the progression or final size of an outbreak, only that the relative sizes of the outbreaks under different scenarios are a meaningful comparison of the effects of different combinations of interventions.}\\[5pt]
\ed{Preliminary observations??}\\[5pt]
We seek to support the activities of the Fall 2020 Planning Task Force via a collaboration with the task force.
}

\usepackage{xcolor}

\newcommand{\ed}[1]{{\color{blue} #1}}

%----------------------------------------------------------------------------------------

\begin{document}

\flushbottom % Makes all text pages the same height

\maketitle % Print the title and abstract box

\tableofcontents % Print the contents section

\thispagestyle{empty} % Removes page numbering from the first page

%----------------------------------------------------------------------------------------
%	ARTICLE CONTENTS
%----------------------------------------------------------------------------------------

\section{intro}

In \ed{March when it became clear that Nova Scotia would be significantly impacted by the novel coronavirus pandemic, Dr. Holger Teismann initiated ....{\tt give some history?}}.
Our team includes three professors in Math \& Stats, \ed{Duane Currie -- the superhero}, and three undergraduate students.
Our primary task is the construction of a set of models (mathematical and computational) which will allow us to simulate COVID-19 on the Acadia campus.
The goal is to use the models to obtain quantitative evidence of the relative importance of a range of possible interventions and policies aimed at preventing, detecting and controlling a possible outbreak on campus.
For example, \ed{is it more effective to convert all classes with more than 50 people to fully online or {\tt something else}}.
Our project is similar to others (see \cite{.....}), but is specifically tailored for the situation at Acadia.

Our primary model is ``agent-based'' and simulates the interaction of all the individuals (around \ed{4700}) on campus.
For students, this includes classes, on-campus residences, the dinning hall, and approximations of other social interactions.
For faculty and staff this includes \ed{what....}


We are currently in the testing and validation stage but have some preliminary observations (see below in Section \ref{sec:prelimresults}).



\section{More details on our agent-based model}
An overview of our model is useful in order to be able to understand the types of experiments we are able to perform and also to help interpret any results from these experiments.

The simulation of our model advances by moving through time in discrete steps (each such time step could represent one day or a specific portion of a day such as six hours).
During each such step each individual is checked for possible interactions with other individuals according to the contact data given in the model.
Each individual in the model is described as being in one of a possible set of states, be it \emph{susceptible}, \emph{infectious}, \emph{quarantined}, \emph{recovered}, etc.
As the simulation advances and interactions occur, individuals potentially (with some probability) transition from one state to another according to the typical progression of COVID-19.
The transition probabilities are calibrated so that the disease dynamics generated by our model corresponds to the generally accepted transmission rates and distribution of time periods (latency period, infectious period, serial interval, etc) for COVID-19.

The possible interactions are governed by a large amount of contact data.
This includes known information about class enrollments, class locations and times, residence room assignments, course instructor assignments, office locations, etc.
We also include reasonable assumptions about social interaction outside of the academic context which is modeled by using standard methods from the research literature (of course, since we cannot have precise information about these non-academic interactions this gives the possibility of some error in our simulations).
Each different category of interaction has a certain level of strength which leads to differing probabilities of disease transmission for the different categories.

\emph{It is important to note that much of this contact data is protected by privacy legislation and so we do not have access to the raw data but only to suitably anonymized versions of the data.}

The different categories of contact data are encoded separately and thus it is possible to modify each of them independently from the others.
For example, it is simple to remove all classes larger than 50 from the ``classroom contact matrix'' and thus measure the effect of moving all such classes from face-to-face instruction to fully online instruction.
It is also possible to modify many other aspects of the simulation.
As an example, should Acadia have weekly access to some number of PCR tests for SARS-CoV-2, we could easily include different testing strategies into the model and experiment with to try to find the most effective way to test (including ``pooled testing'' for screening, should that be possible).

As an idea, a representative (but incomplete) list of possible experimental factors is:
\begin{enumerate}

   \item restricting the size of face-to-face classes

   \item controls on between-class transitions

    \item restricting interaction between residences

    \item cohorting residences by major/program

    \item increased social distancing/mask wearing

    \item testing strategies

    \item contact tracing and quarantine strategies

    \item 


\end{enumerate}
Of course some of these factors can be more exactly implemented in our model than others, but we are happy to explore a range of possibilities.

\ed{what other aspects of the model or possibilities for modification/experimentation do we want to highlight???  Do we want an enumerated list of all the categories of contacts as an explicit indication of what can easily be varied?}




\section{Some preliminary observations from our model simulations}
\label{sec:prelimresults}

\ed{what do we want to include here??}

\section{Future plans}

\ed{a more extensive technical report outlining in detail our models and results}

\ed{more experimentation -- suggest types??}


\section{Questions for the COVID-19 Task Force}


\ed{what do we want to ask them??}

%------------------------------------------------
\phantomsection
\section*{Acknowledgments} % The \section*{} command stops section numbering
\addcontentsline{toc}{section}{Acknowledgments} % Adds this section to the table of contents


%----------------------------------------------------------------------------------------
%	REFERENCE LIST
%----------------------------------------------------------------------------------------
\phantomsection
\bibliographystyle{unsrt}
\bibliography{references}

%----------------------------------------------------------------------------------------

\end{document}